\documentclass[12,french]{report} 
\usepackage{geometry}
\geometry{vmargin=3cm, hmargin=3cm}
\usepackage[T1]{fontenc}
\usepackage[utf8]{inputenc}
\usepackage[french]{babel}
\usepackage{graphicx}
\usepackage{amsmath}
\usepackage{amssymb}
\usepackage{sectsty}
\usepackage{authblk}
\usepackage{algpseudocode}
\usepackage{algorithm}
\usepackage{xspace}
\usepackage{mathtools}
\usepackage{mathrsfs}
\usepackage{enumitem}
\usepackage{titlesec}
\usepackage{hyperref}
\usepackage{xcolor}
\usepackage[justification=centering]{caption}
\usepackage{float}
\usepackage{tabto}

\usepackage{listings}
\usepackage{cleveref}

\renewcommand{\lstlistingname}{Code}
%\renewcommand{\figurename}{Fig.}

\lstdefinestyle{chstyle}{%
backgroundcolor=\color{gray!12},
basicstyle=\ttfamily\small,
showstringspaces=false,
numbers=left}

%\AddThinSpaceBeforeFootnotes
%\FrenchFootnotes

\titleformat{\chapter}[hang]{\bf\Huge}{\thechapter.}{2pc}{}
\titlespacing*{\chapter}{10pt}{0pt}{40pt}[0pt]
\newcommand{\HRule}{\rule{\linewidth}{0.5mm}}

\providecommand{\keywords}[1]{\textbf{\textit{Keywords:}} #1}
\bibliographystyle{apalike}

\usepackage{hyperref}

\begin{document}
\hypersetup{pdfborder=0 0 0}

\begin{titlepage}

\begin{center}
	\vspace*{\stretch{1}}
	\textsc{\LARGE Institut national des sciences appliquées de Rouen} 
	\vspace{5mm}\\
	\includegraphics[width=0.4\textwidth]{./Images/insa}\\[1.0 cm]

	\textsc{\Large Projet MMSN GM3 - Vague 3 - Sujet 4}\\[0.6cm]

	% Title
	\HRule \\[0.5cm]
	{ \Huge \bfseries Etude des erreurs sur la méthode du Gradient Conjugué}\\[0.2cm]
	\HRule \\[0.75cm]

	\includegraphics[width=0.6\textwidth]{./Images/Page_de_garde}\\[0.9 cm]

	% Author and supervisor
	\begin{minipage}{0.4\textwidth}
		\begin{flushleft} \large
			\emph{Auteurs:}\\
			Thibaut \textsc{André-Gallis} \\
			{\small\href{mailto:thibaut.andregallis@insa-rouen.fr}{thibaut.andregallis@insa-rouen.fr}} \\
			Kévin \textsc{Gatel} \\
			{\small\href{mailto:kevin.gatel@insa-rouen.fr}{kevin.gatel@insa-				rouen.fr}}
		\end{flushleft}
	\end{minipage}
	\begin{minipage}{0.4\textwidth}
		\begin{flushright} \large
			\emph{Enseignants:} \\
			Mathieu \textsc{Bourgais} \\
			{\small\href{mailto:mathieu.bourgais@insa-rouen.fr}								{mathieu.bourgais@insa-rouen.fr}}\\
		\end{flushright}
	\end{minipage}
	\vspace*{\stretch{1}}

	\vfill
	{\large 6 Juin 2021}
\end{center}
\end{titlepage}

\tableofcontents

%\listoffigures

\renewcommand{\chaptername}{}
\chapter*{Introduction} %thib
\addcontentsline{toc}{chapter}{Introduction}

\chapter{Présentation du problème} %kev

L'objectif est donc d'étudier les erreurs que fait la machine en utilisant l'arithmétique flottante plutôt que l'ensemble théorique des réels.\\

 Ces erreurs seront étudiées sur la solution du problème linéaire $Ax=b$ avec la méthode du gradient conjugué. En choisissant la matrice $A$ de dimension 4 définie comme ci-dessous :\\

\begin{figure}[H]
	\center
	\includegraphics[width=0.2\textwidth]{./Images/H_4}
	\caption{Matrice de Hilbert de dimension 4}
\end{figure}

Le nombre d'étape pour trouver la solution sera en théorie inférieur ou égale à 4 (assuré par la méthode du gradient conjugué).\\

En notant $K_2(A)$ le conditionnement 2 de A tel que :\footnote{conditionnement obtenu sur Matlab}
$$K_2(a)= 1.5514*10^4$$

On a l'inégalité du conditionnement :
$$\frac{||\Delta x||_{2}}{||x||_2}\leq K_2(A)\frac{||\Delta b||_{2}}{||b||_2}$$

Le test d'arrêt est de la forme $$tol^2*(b,b) > (r,r) $$
avec $(\bullet,\bullet)$ le produit scalaire usuel et $tol=10^{-10}$.\\

Enfin, le vecteur $b$ est choisi comme ci-dessous :

$$b_i=\sum_{k=1}^4a_{ik}$$

de manière à avoir 
$$x^T=\left(\begin{array}{cccc}
1 & 1 & 1 & 1\end{array}\right)$$




\chapter{Vecteur résidu $r$} %thib

\section{Etape 0}

\section{Etape 1}

\section{Etape 2}

\section{Etape 3}

\section{Etape 4}

\chapter{Vecteur solution $x$} %kev

\section{Etape 1}

\section{Etape 2}

\section{Etape 3}

\section{Etape 4}

\chapter{Analyse numérique du problème} %kev

\chapter*{Conclusion} %thib
\addcontentsline{toc}{chapter}{Conclusion}

\chapter*{Annexe}

\end{document}
